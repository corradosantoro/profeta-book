%%
%% preface.tex
%%
\chapter{Preface}
%%
{\raggedleft{}\em%%
``That the sperm of a man be putrefied by itself in a sealed cucurbit for\\
forty days with the highest degree of putrefaction in a horse's womb, or at\\
least so long that it comes to life and moves itself, and stirs, which is
easily observed.\\
%%
After this time, it will look somewhat like a man, but
transparent, without a body.\\
If it be fed wisely with
human blood, and be nourished for up to forty weeks,\\
and be kept
in the even heat of the horse's womb, a living human child grows therefrom,\\
with all its members like another child, which is born of a woman, but much
smaller''\\
Paracelsus---De Natura Rerum~\cite{homunculus}\\}
%%
%%
\section{The Dream of the ``Golem'':\\from Alchimists to Computer Scientists}
%%
Creating \emph{artificial autonomous systems} is something that men have
always tried to do.
%%
According to the hebrew tradition, the knowledge of the \emph{Quabbalah}
allows a human to create, from the clay, a giant that is strong and
submissive, and able to perform hard works (like a slave) on behalf of its
creator.
%%
Indeed, from the impossible recipes of the alchimist \emph{Paracelsus} to create
the \emph{homunculus}~\cite{homunculus}, till the \emph{mechanical toys} of
'700 and '800, like the three automata of Pierre Jaquet Droz (the
\emph{musicist}, the \emph{drawer} and the
\emph{writer})~\cite{jaquet-droz}, or the
mechanical ``duck'' of Jacques De
Vaucanson\footnote{Jacques De Vaucanson~\cite{de-vaucaunson} was famous for
  the invention of the
  first automatic loom which is also considered as the first programmable
  machine},
one of the main
dreams of people and
scientists has been the creation of artificial life, with a particular
focus on ``systems that can think''.
%%
Such an aspect has been particularly stressed when computer systems
appeared, first by making some theoretical assumptions (as in the Turing
Test), and then by concretely trying to emulate certain aspects as in the
\emph{Artificial Intelligence Era}.
%%
But despite of claims or expectations, till nowadays we could not
implement an
artificial system that is able to ``think like a human''; instead, we are able
to emulate certain specific mechanisms of human life~\cite{RusselNorvig}:
expert systems can emulate reasoning (to a certain extent), artificial neural
networks are able to learn
processes or patterns, and genetic algorithms can find ``optimal
solutions'' to mathematical problems by using the same principles of the
natural selection.
%%

%%
To support the cited (and other AI) techniques, computer scientists
concentrated not only on the basic mechanism to be emulated in an algorithm
but also on the opportunity to use or introduce \emph{programming
  languages} that provide constructs and semantics better fitting the
implementation of the problem to be solved.
%%
This is the reason why programming languages like Prolog, LISP or Scheme
appeared: they present features that are not available in classical
imperative approaches, thus making easier the development of ``intelligent
systems''.
%%

%%
Among these alternative languages, in the particular applications in which
a certain form of ``reasoning''
is required, \emph{logic/declarative approaches} clearly proved their
validity.
%%
They are based on a model strongly exploiting the concept of
\emph{knowledge} which can come from both some pre-constituted \emph{facts}
(initial knowledge)
and some \emph{inference rules} that are able to \textbf{derive new
knowledge}.
%%
In these approaches, the semantics of the inference rules depends on the
specific reasoning model adopted, which ranges from simple
\emph{first-order logic} predicates, to \emph{enhanced first-order logic}
(or higher-order logics), \emph{fuzzy logic}, \emph{temporal logic}, etc.
%%
%%


%%
\section{Situated AI systems: Agents}
%%
As an evolution of AI systems, the '90s have seen the birth of special kind
of autonomous computer systems: \emph{agents}~\cite{bradshaw};
they are basically pieces of software able to act autonomously, that live in
a specific environment (virtual of physical), and with the task of
achieving a specific goal without any external (human) intervention.
%%
With respect to traditional AI systems, agents include the characteristic of
\emph{``situatedness''}, i.e.~their operations are strongly tied to the
environment in which they live and, for this reason, they must be able to
perform proper interactions with such an environment.
%%
On this basis, agents stress the aspect of \emph{interaction} with both the
environment and other agents (as in the case of \emph{multi-agent
  systems}~\cite{weiss}): software architectures, languages and frameworks
currently available for agent-based systems strongly take into account such
an aspect, sometimes giving a higher importance with respect to the
intelligence, but in any case letting interaction to play a fundamental
role.
%%
This change in focus, together with the widespread diffusion that the
Java\texttrademark{}
language had since '90s, increased the development of agent-based
software and platforms that use this language.
%%
However, while Java\texttrademark{}, being imperative and object-oriented
in its nature,
could be adequate for interaction or to program an autonomous behaviour (as
in the case of the well-known platform JADE~\cite{jade-book}), it is not
suitable to implement a form of reasoning but additional external tools
must be integrated~\cite{jess,drools,jasonbook}.
%%


%%
\section{Physical Agents: the Robots}
%%
While agents are software entities often living in a virtual environment,
there is a more interesting category of autonomous systems characterized by
a ``physical nature'': \emph{robots} are ``real creatures'' that live in
the real physical world sharing with humans all the aspects of the
environment in which all of them live.
%%
Programming \emph{autonomous robots} is more, more and more challenging
than programming agents, since the interaction with the environment poses
problems that are not present in agent-based applications.
%%
Indeed, the most important difference relies on the fact that a physical
environment is \emph{strongly unpredictable}, meaning that the probability
that an action made by the robot \emph{fails} is extremely high; and a
failure, if not properly handled, implies the impossibility to achieve the
goal for which the robot has been designed.
%%
Managing
unpredictability and failures means, at first sight, the ability to
detect---in any possible state---the events in the environment that could
impede the execution of a certain action, and, secondly, the ability of
find possible alternative strategies---if any---to try to reach that goal.
%%
These considerations, from the programming language/approach perspective, imply
\emph{(i)} that the reasoning aspect must be tightly mingled with the way in which
the environment is sensed and \emph{(ii)} that \emph{all possible events} must be
always sensed (the robot must never be ``blind'') and properly handled in
any possible situation.
%%


%%
\section{... and that's why PROFETA}
%%
All of the said principles and characteristics have been analyzed and
taken into account in the implementation of some autonomous robots in the
ARSLAB of the University of Catania.
%%
As a result, the PROFETA tool has been designed with the objective of
facing all the said aspect in a flexible way for a programmer.
%%
The name stands for \emph{Python RObotic Framework for
  dEsigning sTrAtegies} and is a Python-based tool composed of the
following parts:
%%
\begin{itemize}
%%
\item A \textbf{knowledge system}, to let a programmer representing the
  robot's knowledge by means of some \emph{facts} (which indeed are called
  \emph{beliefs}), properly stored and managed through a \emph{knowledge
    base}.
%%
\item A set of \textbf{sensor} and \textbf{action} classes, provided as
  \emph{artifacts} to perform the interaction with the environment.
%%
\item A \textbf{declarative language} that allows a programmer to express the
  robot behaviour by means a rational approach, also exploiting beliefs in
  order to perform a---more or less flexible---form of reasoning.
%%
\end{itemize}
%%
This book describes PROFETA, by presenting its underlying model, the basic
characteristics and the way in which it can be used to write
the behaviour of a robot.
%%
The approach we use in the whole book is mainly practical, all the features
will be presented by providing several examples in order to let the reader
to fully understand the explained characteristic.
%%


%%
\section{Acknowledgements}
%%
And before starting to deal with all the details of PROFETA, it is
mandatory to thank all the people that, in several forms, contributed to
the development of PROFETA with both concrete help (implementation of
parts) and advices or debugging.
%%
They are: Loris Fichera, Daniele Marletta, Vincenzo (Enzo) Nicosia,
Andrea ``Mancausoft'' Milazzo, Carlo Battiato, Riccardo Massari, Fabrizio
Messina, Giuseppe Pappalardo.
%%

