%%
%% basics.tex
%%
\chapter{First Steps with PROFETA}
%%
In this Chapter we will start to familiarize with PROFETA.
%%
We will initially understand the principles of the model behind PROFETA, the
\emph{Belief-Desire-Intention}, and subsequently we'll begin to use the
tool, first by learning how to install it and then by trying to implement
few simple examples.
%%
This will help us to understand, step-by-step, the syntax and semantics of
the PROFETA language which will be then deeply explained in the other
Chapters.
%%
Since PROFETA is written in Python, a quite good knowledge of the language
is mandatory to better understand what is described in this book.
%%

%%
In this Chapter, we consider the use of Linux and Bash; this applies also
to MacOs, but if you
have a different operating system (e.g.~Windows), you need to properly
adapt the commands to your environment, however source codes remain the
same.
%%
In showing the examples, in the snapshots of shell windows, we will use
\underline{underlining} to indicate a command that has to be typed by the
user.
%%

%%


%%
\section{PROFETA Phylosophy: Belief-Desire-Intention}
%%
PROFETA is a tool designed to write autonomous systems (\emph{agents} or
\emph{robots}) exploiting on an
agent-based paradigm called
\emph{Belief-Desire-Intention (BDI)}~\cite{rao1995bdi} and inspired to a
kernel language called AgentSpeak(L)~\cite{AgentSpeak}.
%%
The BDI paradigm is an extension of PRS\footnote{Practical Reasoning
  Systems~\cite{PRSarchitecture}}, based on some principles that aim at
mimicking human thinking.
%%
As the name suggests, it considers three main entities \emph{Beliefs},
\emph{Desires} and \emph{Intentions}.
%%

%%
\textbf{\emph{Beliefs}} (also called \emph{facts} in the AI world)
represent the
\emph{knowledge} of the autonomous system and
are used to reflect both the state of the environment and the additional
knowledge eventually derived by the system.
%%

%%
\textbf{\emph{Desires}} represent the motivational state of the system or,
in other words, the ``things to do'' to achieve its \emph{goals}.
%%
Desires are expressed as \emph{plans} (i.e.~once again
``things to do'') which can be selected for execution following the
occurrence of certain
events in the environment and provided that the system has a specific
knowledge.
%%
From the computer programming point of view, plans of a BDI system
represents the \emph{program} to be executed.
%%

%%
While desires are the overall plans that make it possible the achievement
of goals, \textbf{\emph{Intentions}} represent the \emph{plans selected for
execution}, at a certain time instant, to face a certain situation.
%%
Intentions are indeed considered as \emph{run-time instantiation} of plans.
%%

%%
BDI systems are usually controlled by a virtual machine executing a loop
which performs the following operations:
%%
\begin{itemize}
%%
\item Sample the environment and update the knowledge base with the new
  \emph{beliefs}
%%
\item Identify the \emph{plans} (\emph{desires})
  that, on the basis of the current system status, can be selected for
  execution
%%
\item Selected plans become \emph{intentions} and they can be scheduled for
  execution
%%
\item Intentions are finally executed
%%
\end{itemize}
%%


%%
\section{PROFETA Installation}
%%
The PROFETA tool is entirely written in Python\footnote{PROFETA does not
  support Python3}  and thus depends only on it,
no additional package or library is needed.
%%
It can be downloaded from \texttt{github} using the
\texttt{git clone} command.
%%
To this aim, the git tool must be installed in your system.
%%

%%
After typing:
%%
\shellsnap{\$ \underline{git clone http://github.com/corradosantoro/profeta}}
%%
a folder \texttt{profeta} will appear in your current directory.
%%
In order to install the tool just type:
%%
\shellsnap{%%
\$ \underline{cd profeta}\\
\$ \underline{sudo python setup.py install}}
%%
If no errors occur (the system will ask you for the super-user password)
the PROFETA tool will be installed in your system.
%%
%%
You can now test your installation by running the \texttt{factorial} sample
program:
\shellsnap{%%
\$ \underline{cd samples}\\
\$ \underline{python factorial.py}\\
Enter +fact(N) to calculate the factorial of N\\
PROFETA>}
%%
The prompt which will be displayed is the \emph{PROFETA shell}, that
provides some commands to manipulate the knowledge base, to execute a
goal, etc.
%%
The details of its usage will be given in Chapter~\ref{xxxx}.
%%
To test sample program, just type \texttt{+fact(10)} and the system will
give the answer by
printing the factorial of 10:
%%
\shellsnap{%%
PROFETA>\underline{+fact(10)}\\
PROFETA>the resulting factorial is 3628800}
%%
Type \texttt{quit} to return to the system shell.
%%


%%
\section{``Hello World'' from PROFETA}
%%
As it happens when you start to learn any other programming language, also
for PROFETA the first example we provide is the classical ``Hello world''.
%%
What we intend to do is a reactive program that prints the string
\texttt{'Hello world' from PROFETA} as soon as a specified belief is
asserted in the knowledge base.
%%
The source code of this program is reported in
Figure~\ref{fig:hello} and explained in the following.
%%

%%
\begin{figure}[b!]
\python%%
\begin{lstlisting}
from profeta.lib  import *
from profeta.main import *

# a "say" belief
class say(Belief): pass

PROFETA.start()  # instantiate the engine

# define a rule: when you say "hi", PROFETA answers "hello world"
+say("hi") >> [ show_line("'Hello world' from PROFETA") ]

PROFETA.run_shell(globals())
\end{lstlisting}
\caption{``Hello word'' in PROFETA}\label{fig:hello}
\end{figure}
%%
%%


%%
Lines 1--2 report the basic PROFETA library imports: use them in any
PROFETA-based program.
%%
Line 5 is the \textbf{declaration of a belief}; like in a programming
language, like C/C++ or Java, you need to declare your variables with types, also in
PROFETA you are asked to declare your beliefs (and also the other entities
then you will use).
%%
Belief declaration is done by simply \emph{subclassing} the basic
\texttt{Belief} class.
%%
%%
Line 7 starts the PROFETA engine; this statement is needed before
the PROFETA program appears.
%%
%%
Line 10 reports the ``real'' PROFETA program: in particular, the line
expresses that ``when the belief \texttt{say("hi")} is asserted, the
message \texttt{'Hello world' from PROFETA} is printed on the console.
%%

%%
Line 10 is a \textbf{reactive plan}: \texttt{say("hi")} is a \emph{belief
  pattern}, while the symbol ``\texttt{+}'' completes the pattern by
specifying ``addition'', that is, belief assertion.
%%
Symbol ``\verb+>>+'' plays the role of ``implies'' and its right-hand side
is a \emph{list of actions} to be executed following that assertion; in our
case, the list is composed of a single (library) action which is the output
a message.
%%
%%
Finally, line 12 starts the PROFETA Shell so that we can manually trigger
the defined plan by using the shell commands.
%%

%%
In order to test the ``Hello world'' program, we can just type the commands
displayed in the window below:
%%
\shellsnap{%%
\$ \underline{python hello\_world.py}\\
PROFETA>\underline{+say("hi")}\\
PROFETA>'Hello world' from PROFETA\\
\underline{quit}\\
\$
}
%%


%%
\section{Playing with Beliefs and the Knowledge Base}
%%
In the previous Section, we used the shell command \texttt{+say("hi")} to
trigger a plan.
%%
Indeed, plan triggering is a \emph{secondary effect} of that command since
basically it \textbf{adds} belief \texttt{say("hi")} to the knowledge base;
this effect can be observed by using the shell command \texttt{kb} which
prints out the content of the knowledge base:
%%
\shellsnap{%%
\$ \underline{python hello\_world.py}\\
PROFETA>\underline{+say("hi")}\\
PROFETA>'Hello world' from PROFETA\\
\underline{kb}\\
{[say('hi')]}\\
PROFETA>}
%%

%%
Now try to assert some other beliefs, e.g.~\texttt{say("hello")},
\texttt{say("ciao")}, \texttt{say("bonjour")} and inspect once again the
content of the knowledge base. You now will see the KB populated by the
beliefs you've asserted, but, since none of that beliefs correspond to the
pattern of the plan we defined in the running program, that plan will be
never triggered:
%%
\shellsnap{%%
PROFETA>\underline{+say("hello")}\\
PROFETA>\underline{+say("ciao")}\\
PROFETA>\underline{+say("bonjour")}\\
PROFETA>\underline{kb}\\
{[say('hi'), say('hello'), say('ciao'), say('bonjour')]}\\
PROFETA>}
%%

%%
You can now try to assert again belief \texttt{say("hi")} and see what
happens:
%%
\shellsnap{%%
...\\
PROFETA>\underline{kb}\\
{[say('hi'), say('hello'), say('ciao'), say('bonjour')]}\\
PROFETA>\underline{+say("hi")}\\
PROFETA>}
%%

%%
Indeed \emph{nothing happens!} This is quite normal and in accordance with
beliefs semantics because re-asserting a yet existing belief does not
provoke any change in the knowledge base and thus no related event is
generated.
%%
But if you want to restart the plan, you can \textbf{remove} the belief by
using the command \texttt{-say("hi")}, and assert it again:
%%
\shellsnap{%%
...\\
PROFETA>\underline{-say("hi")}\\
PROFETA>\underline{kb}\\
{[say('hello'), say('ciao'), say('bonjour')]}\\
PROFETA>\underline{+say("hi")}\\
PROFETA>'Hello world' from PROFETA\\
\underline{kb}\\
{[say('hello'), say('ciao'), say('bonjour'), say('hi')]}\\
PROFETA>}
%%

%%
\section{Belief Syntax and Semantics}
%%
After having practically seen how beliefs are handled and when then can
generate events, let's make a complete discussion about their syntax and
semantics.
%%

%%
A \textbf{belief} $B$ in PROFETA is characterised by a \emph{predicate}
(i.e.~the belief name) and zero or more parameters.
%%
It is declared as a Python class that simply extends the basic library
class \texttt{Belief} and is referred as classical logic \emph{atomic
  formula} with zero or more \emph{ground terms}, i.e.~$B(t_1, \dots, t_n)$
with $n \ge 0$ and $t_1, \dots, t_n$ as constants.
%%

%%
\emph{Asserting} a belief $B(t_1, \dots, t_n)$ implies to add it to the
knowledge base of your PROFETA program, given that \emph{the same belief
  with the same parameters} does not yet exist in the KB, otherwise
adding is not performed\footnote{There are two exceptions to this rule
  which apply when different kind of beliefs are used, that are
  \texttt{SingletonBelief}s and \texttt{Reactor}s; their role and semantics
  will be described in Chapter~\ref{yyy}.}.
%%
From the reasoning perspective, this operation represents the case in which
the agent \emph{knows} the fact (i.e.~thinks that the fact is
\textbf{true}) that is
represented by that belief.
%%
Assert operation (when succeeds) also generate an \emph{adding event} that
can be caught by a plan in order to execute some computations.
%%

%%
Similarly, \emph{retracting} a belief  $B(t_1, \dots, t_n)$ implies to
remove it from the
knowledge base of your PROFETA program, given that the same belief
(with the same parameters) \emph{already exists} (in the KB), otherwise
removal cannot be performed.
%%
Also retract operation, when it succeeds, generates a relevant event
(\emph{delete event}) that can be used to trigger a plan, as it will be
shown in the examples reported in the next Sections.
%%

