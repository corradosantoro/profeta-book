%%
%% basics.tex
%%
\chapter{First Steps with PROFETA}
%%
In this Chapter we will start to familiarize with PROFETA.
%%
We will initially understand the principles of the model behind PROFETA, the
\emph{Belief-Desire-Intention}, and subsequently we'll begin to use the
tool, first by learning how to install it and then by trying to implement
few simple examples.
%%
This will help us to understand, step-by-step, the syntax and semantics of
the PROFETA language which will be then deeply explained in the other
Chapters.
%%
Since PROFETA is written in Python, a quite good knowledge of the language
is mandatory to better understand what is described in this book.
%%

%%
In this Chapter, we consider the use of Linux and Bash; this applies also
to MacOs, but if you
have a different operating system (e.g.~Windows), you need to properly
adapt the commands to your environment, however source codes remain the
same.
%%
In showing the examples, in the snapshots of shell windows, we will use
\underline{underlining} to indicate a command that has to be typed by the
user.
%%

%%


%%
\section{PROFETA Phylosophy: Belief-Desire-Intention}
%%
PROFETA is a tool designed to write autonomous systems (\emph{agents} or
\emph{robots}) exploiting an
agent-based paradigm called
\emph{Belief-Desire-Intention (BDI)}~\cite{rao1995bdi} and inspired to a
kernel language called AgentSpeak(L)~\cite{AgentSpeak}.
%%
The BDI paradigm is an extension of PRS\footnote{Practical Reasoning
  Systems~\cite{PRSarchitecture}}, based on some principles that aim at
mimicking human thinking.
%%
As the name suggests, it considers three main entities \emph{Beliefs},
\emph{Desires} and \emph{Intentions}.
%%

%%
\textbf{\emph{Beliefs}} (also called \emph{facts} in the AI world)
represent the
\emph{knowledge} of the autonomous system and
are used to reflect both the state of the environment and the additional
knowledge eventually derived by the system.
%%

%%
\textbf{\emph{Desires}} represent the motivational state of the system or,
in other words, the ``things to do'' to achieve its \emph{goals}.
%%
Desires are expressed as \emph{plans} (i.e.~once again
``things to do'') which can be selected for execution, following the
occurrence of certain
events in the environment and provided that the system has a specific
knowledge.
%%
From the programming point of view, plans of a BDI system
represents the \emph{program} to be executed.
%%

%%
While desires are the overall plans that make it possible the achievement
of goals, \textbf{\emph{Intentions}} represent the \emph{plans selected for
execution}, at a certain time instant, to face a certain situation.
%%
Intentions are indeed considered as \emph{run-time instantiation} of plans.
%%

%%
BDI systems are usually controlled by an abstract machine executing a loop
which performs the following operations:
%%
\begin{itemize}
%%
\item Sample the environment and update the knowledge base with the new
  \emph{beliefs};
%%
\item Identify the \emph{plans} (\emph{desires})
  that, on the basis of the current sets of beliefs, can be selected for
  execution;
%%
\item Selected plans become \emph{intentions} and they can be scheduled for
  execution;
%%
\item Intentions are finally executed.
%%
\end{itemize}
%%


%%
\section{PROFETA Installation}
%%
The PROFETA tool is entirely written in Python (version 2)  and thus
depends only on it\footnote{PROFETA does not
  support Python3},
no additional package or library are needed.
%%
It can be downloaded from \texttt{github} using the
\texttt{git clone} command.
%%
To this aim, the \texttt{git} tool must be installed in your system.
%%

%%
After typing:
%%
\shellsnap{\$ \underline{git clone http://github.com/corradosantoro/profeta}}
%%
a folder \texttt{``profeta''} will appear in your current directory.
%%
In order to install the tool just type:
%%
\shellsnap{%%
\$ \underline{cd profeta}\\
\$ \underline{sudo python setup.py install}}
%%
If no errors occur (the system will ask you for the super-user password)
the PROFETA tool will be installed in your system.
%%
%%
You can now test your installation by running the \texttt{factorial} sample
program:
\shellsnap{%%
\$ \underline{cd samples}\\
\$ \underline{python factorial.py}\\
Enter +fact(N) to calculate the factorial of N\\
PROFETA>}
%%
The prompt which will be displayed is the \emph{PROFETA shell}, that
provides some commands to manipulate the knowledge base, to execute a
goal, etc.
%%
The details of its usage will be given in Chapter~\ref{xxxx}.
%%

%%
To test sample program, just type \texttt{+fact(10)} and the system will
give the answer by
printing the factorial of 10:
%%
\shellsnap{%%
PROFETA>\underline{+fact(10)}\\
PROFETA>the resulting factorial is 3628800}
%%
Type \texttt{quit} to return to the system shell.
%%


%%
\section{``Hello World'' from PROFETA}
%%
As it happens when you start to learn any other programming language, also
for PROFETA the first example we provide is the classical ``Hello world''.
%%
What we intend to do is a reactive program that prints the string
\texttt{'Hello world' from PROFETA} as soon as a specified belief is
asserted in the knowledge base.
%%
The source code of this program is reported in
Figure~\ref{fig:hello} and explained in the following.
%%

%%
\begin{figure}[b!]
\python%%
\begin{lstlisting}
from profeta.lib  import *
from profeta.main import *

# declaration of "say" belief
class say(Belief): pass

PROFETA.start()  # instantiate the engine

# define a rule: when you say "hi", PROFETA answers "hello world"
+say("hi") >> [ show_line("'Hello world' from PROFETA") ]

PROFETA.run_shell(globals())
\end{lstlisting}
\caption{``Hello world'' in PROFETA}\label{fig:hello}
\end{figure}
%%
%%


%%
Lines 1--2 report the basic PROFETA library imports: use them in any
PROFETA-based program.
%%
Line 5 is the \textbf{declaration of a belief}; like in a programming
language, like C/C++ or Java, you need to declare your variables with types, also in
PROFETA you are asked to declare your beliefs (and also the other entities
then you will use).
%%
Belief declaration is done by simply \emph{subclassing} the basic
\texttt{Belief} class.
%%
%%
Line 7 starts the PROFETA engine; this statement is needed before
the PROFETA program appears.
%%

%%
Line 10 reports the ``real'' PROFETA program: in particular, the line
expresses that ``when the belief \texttt{say("hi")} is asserted, the
message \texttt{'Hello world' from PROFETA} is printed on the console.
%%
%%
Line 10 is a \textbf{reactive plan}: \texttt{say("hi")} is a \emph{belief
  pattern}, while the symbol ``\texttt{+}'' completes the pattern by
specifying ``addition'', that is, belief assertion.
%%
Symbol ``\verb+>>+'' plays the role of ``implies'' and its right-hand side
is a \emph{list of actions} to be executed following that assertion; in our
case, the list is composed of a single (library) action which is the output
of the message.
%%
%%
Finally, line 12 starts the PROFETA Shell so that we can manually trigger
the defined plan by using the shell commands.
%%

%%
In order to test the ``Hello world'' program, we can just type the commands
displayed in the window below:
%%
\shellsnap{%%
\$ \underline{python hello\_world.py}\\
PROFETA>\underline{+say("hi")}\\
PROFETA>'Hello world' from PROFETA\\
\underline{quit}\\
\$
}
%%


%%
\section{Playing with Beliefs and the Knowledge Base}
%%
In the previous Section, we used the PROFETA shell command
\texttt{+say("hi")} to
trigger a plan.
%%
Indeed, plan triggering is a \emph{secondary effect} of that command since
basically it \textbf{adds} the belief \texttt{say("hi")} to the knowledge base;
this behaviour can be observed by using the shell command \texttt{kb} which
prints out the content of the knowledge base:
%%
\shellsnap{%%
\$ \underline{python hello\_world.py}\\
PROFETA>\underline{+say("hi")}\\
PROFETA>'Hello world' from PROFETA\\
\underline{kb}\\
{[say('hi')]}\\
PROFETA>}
%%

%%
Now try to assert some other beliefs, e.g.~\texttt{say("hello")},
\texttt{say("ciao")}, \texttt{say("bonjour")} and inspect once again the
content of the knowledge base. You now will see the KB populated by the
beliefs you've asserted, but, since none of that beliefs correspond to the
pattern of the plan we defined in the running program, that plan will be
never triggered:
%%
\shellsnap{%%
PROFETA>\underline{+say("hello")}\\
PROFETA>\underline{+say("ciao")}\\
PROFETA>\underline{+say("bonjour")}\\
PROFETA>\underline{kb}\\
{[say('hi'), say('hello'), say('ciao'), say('bonjour')]}\\
PROFETA>}
%%

%%
You can now try to assert again belief \texttt{say("hi")} and see what
happens:
%%
\shellsnap{%%
...\\
PROFETA>\underline{kb}\\
{[say('hi'), say('hello'), say('ciao'), say('bonjour')]}\\
PROFETA>\underline{+say("hi")}\\
PROFETA>}
%%

%%
Indeed \emph{nothing happens!} This is quite normal and in accordance with
beliefs semantics because re-asserting a yet existing belief does not
provoke any change in the knowledge base and thus no related event is
generated.
%%
But if you want to restart the plan, you can \textbf{remove} the belief by
using the command \texttt{-say("hi")}, and assert it again:
%%
\shellsnap{%%
...\\
PROFETA>\underline{-say("hi")}\\
PROFETA>\underline{kb}\\
{[say('hello'), say('ciao'), say('bonjour')]}\\
PROFETA>\underline{+say("hi")}\\
PROFETA>'Hello world' from PROFETA\\
\underline{kb}\\
{[say('hello'), say('ciao'), say('bonjour'), say('hi')]}\\
PROFETA>}
%%

%%
\section{Belief Syntax and Semantics}
%%
After having practically seen how beliefs are handled and when then can
generate events, let's make a complete discussion about their syntax and
semantics.
%%

%%
A \textbf{belief} $B$ in PROFETA is characterised by a \emph{predicate}
(i.e.~the belief name) and zero or more parameters.
%%
It is declared as a Python class that simply extends the basic library
class \texttt{Belief} and is referred as classical logic \emph{atomic
  formula} with zero or more \emph{ground terms}, i.e.~$B(t_1, \dots, t_n)$
with $n \ge 0$ and $t_1, \dots, t_n$ as constants.
%%

%%
\emph{Asserting} a belief $B(t_1, \dots, t_n)$ implies to add it to the
knowledge base of your PROFETA program, given that \emph{the same belief
  with the same parameters} does not yet exist in the KB, otherwise
adding is not performed\footnote{There are two exceptions to this rule
  which apply when different kind of beliefs are used, that are
  \texttt{SingletonBelief}s and \texttt{Reactor}s; their role and semantics
  will be described in Chapter~\ref{yyy}.}.
%%
From the reasoning perspective, this operation represents the case in which
the agent \emph{knows} the fact (i.e.~thinks that the fact is
\textbf{true}) that is
represented by that belief.
%%
Assert operation (when succeeds) also generate an \emph{adding event} that
can be caught by a plan in order to execute some computations.
%%

%%
Similarly, \emph{retracting} a belief  $B(t_1, \dots, t_n)$ implies to
remove it from the
knowledge base of your PROFETA program, given that the same belief
(with the same parameters) \emph{already exists} (in the KB), otherwise
removal cannot be performed.
%%
Also retract operation, when it succeeds, generates a relevant event
(\emph{delete event}) that can be used to trigger a plan, as it will be
shown in the examples reported in the next Sections.
%%


%%
\section{Using Variables}
%%
Let us imagine that we want to generalise the \texttt{hello\_world} program
by handling any type of message, rather than only \texttt{"hi"}.
%%
We can modify the program as reported in Figure~\ref{fig:hello_gen},
call it simply \texttt{hello.py}, and run it as follows:
%%
\shellsnap{%
\$ \underline{python hello.py}\\
PROFETA>\underline{+say("ciao")}\\
PROFETA>you said ciao and I say 'Hello world' from PROFETA\\
\underline{+say("hello")}\\
PROFETA>you said hello and I say 'Hello world' from PROFETA\\
\underline{+say("hi")}\\
PROFETA>you said hi and I say 'Hello world' from PROFETA\\
\underline{kb}\\
{[say('ciao'), say('hello'), say('hi')]}\\
PROFETA>}
%%

%%
\begin{figure}[b!]
\python%%
\begin{lstlisting}
from profeta.lib  import *
from profeta.main import *

# declaration of "say" belief
class say(Belief): pass

PROFETA.start()  # instantiate the engine

+say("X") >> [ show_line("you said","X",
                         "and I say 'Hello world' from PROFETA") ]

PROFETA.run_shell(globals())
\end{lstlisting}
\caption{Generalised ``Hello world'' in PROFETA}\label{fig:hello_gen}
\end{figure}
%%
%%

%%
The difference of this program, with respect to the \texttt{hello\_world}
is that now \emph{any parameter} given when asserting belief \texttt{say()}
is able to trigger the plan; moreover, that parameter is also printed in the
message shown by the plan itself.
%%
This is achieved by using a \emph{variable}, \texttt{X}, which is
bound to the value present in belief assertion and then used also in the
printed message.
%%
The example shows how to use variables in PROFETA plans: as in a
Prolog-like style, \emph{any string parameter starting with an uppercase letter}
is treated as \emph{variable} and used according to the styles of logic
programming.
%%
Scope of variables is per-plan, meaning that \emph{(i)} a variable exists
only for the
duration of plan execution and  \emph{(ii)}  two variables having the same
name, but used in different plans, are \emph{different} as well.
%%


%%
\section{Adding Conditions}
%%
PROFETA plans are triggered by events that, according to what we have seen
till now, can be the assertion or retracting of a belief, also with specific
patterns in the parameters.
%%
In order to allow a better flexibility, plans can also include a
\textbf{condition}, represented by a logic predicate that, if true, make it
possible the execution of the plan itself.
%%
Therefore, in this case, not only the triggering event must
occur but also the condition part must be true.
%%
The condition allows a programmer to ensure that certain
beliefs  (also with given parameters) are present in the knowledge base and
that the (bound) variables used in the plan have specific values.
%%
As in Prolog clause, variables used in the event and in the condition are
\emph{unified}.
%%
From the syntactical point of view, the condition is expressed using the
\emph{``/''} operator placed immediately after the event.
%%
The following example will make condition syntax and semantics more clear.
%%

%%
Let us suppose we want to represent a classroom of students that, at a
certain time, take the university degree.
%%
We use the belief \texttt{student(X)} to state that \texttt{X} is a student
and  \texttt{graduated(X)} to state that \texttt{X} is now graduated.
%%
Obviously, these two beliefs obey to two main constraint:
%%
\begin{description}
%%
\item[{[C1]}] They are mutually exclusive, that is, a person is either a student or
  a graduated;
%%
\item[{[C2]}] Before being graduated, a person must have been a student;
%%
\item[{[C3]}] A graduated person cannot become again a student (ynless a
  different faculty is chosen, but we don't consider here such a case).
%%
\end{description}
%%

%%
These constraints can be expressed as proper plans to be triggered by the
assertion of those beliefs.
%%
In particular:
%%
\begin{description}
%%
\item[{[P1]}] When belief \texttt{graduated(X)} is asserted and \texttt{student(X)}
  is also present in the KB, the second belief must be removed (indicating
  that \texttt{X} became graduated, constraints \textbf{C1} and
  \textbf{C2}).
%%
\item[{[P2]}] When belief \texttt{graduated(X)} is asserted and \texttt{student(X)}
  is \textbf{not present} in the KB, the first belief must be removed (it's
  a violation of constraint \textbf{C2}).
%%
\item[{[P3]}] When belief \texttt{student(X)} is asserted it must be
immediately removed if also \texttt{graduate(X)} is present in the KB
(constraints \textbf{C1} and \textbf{C3}).
%%
\end{description}
%%

%%
\begin{figure}[b!]
\python%%
\begin{lstlisting}
from profeta.lib  import *
from profeta.main import *

# the Beliefs
class student(Belief): pass
class graduated(Belief): pass

PROFETA.start()

+graduated("X") / student("X") >> \
    [ -student("X"),
      show_line("congratulations", "X", "is now graduated") ]
+graduated("X") >> \
    [ -graduated("X"),
      show_line("cannot graduate", "X", "; it is not a student") ]
+student("X") / graduated("X") >> \
    [ -student("X"),
      show_line("X", "is graduated and cannot become a student") ]

PROFETA.run_shell(globals())
\end{lstlisting}
\caption{The ``graduating students'' program}\label{fig:student}
\end{figure}
%%
%%

%%
The listing of the PROFETA program \texttt{student.py}, behaving as
described, is reported in Figure~\ref{fig:student}.
%%
Here, we can see how conditions (also together with variable unification)
can be used in PROFETA\footnote{The \texttt{``\textbackslash''} is used
in Python to break lines of an expression as in the listing.}.
%%

%%
Lines 10--12 represents the plan \textbf{P1}: it is triggered by the
assertion of \texttt{graduated(X)} but only if also \texttt{student(X)} is
present in the knowledge base; as a consequence, the latter belief is
removed and a proper message is displayed.
%%

%%
When a \texttt{graduated(X)} belief is asserted, if the condition of the
previous plan is false (the \texttt{student(X)} belief is not asserted),
the PROFETA engine analyses the next plan (lines
13--14) and executes it; this corresponds to \textbf{P2} whose action
is the removal of \texttt{graduated(X)} since it would be inconsistent.
%%

%%
The last plan (lines 16--18, corresponding to \textbf{P3}) is triggered
when belief \texttt{student(X)} is asserted by the student is already
graduated: also in this case there is a consistency violation and, as a
result, the asserted belief must be removed.
%%

%%
An sample run of the program shown can be seen in the following window:
%%
\shellsnap{%
\$ \underline{python students.py}\\
PROFETA>\underline{+student("corrado")}\\
PROFETA>\underline{+student("alice")}\\
PROFETA>\underline{+graduated("john")}\\
PROFETA>cannot graduate john , it is not a student\\
\underline{+student("john")}\\
PROFETA>\underline{kb}\\
{[student('corrado'), student('alice'), student('john')]}\\
PROFETA>\underline{+graduated("alice")}\\
PROFETA>congratulations alice is now graduated\\
\underline{kb}\\
{[student('corrado'), student('john'), graduated('alice')]}\\
PROFETA>\underline{+student("alice")}\\
PROFETA>alice is graduated and cannot become a student again\\
PROFETA>}
%%


%%
\section{More Complex Conditions}
%%
We have learned how to add conditions on the basis of the presence of
certain beliefs in the KB, however conditions can be also used to test the
value of the variables used in the plan: this is achieved by using Python
\emph{lambda expressions} that evaluate to a boolean value.
%%
In order to understand this, let us implement a specific example program:
the \emph{Sieve of Eratosthenes}.
%%
Prime numbers generation is one of the most widely used examples in
explaining or presenting the features of a programming language and, on
this basis, different implementation philosophies are proposed.
%%
In our PROFETA implementation we will use the following schema.
%%
\begin{enumerate}
%%
\item Let first generate all numbers from 2 to 100;
%%
\item Then, find all possible pairs $(X,Y)$ and ``drop'' $X$ if it is a
  multiple of $Y$;
%%
\item When no more ``drops'' are possible, the remaining numbers are
  \emph{prime}.
%%
\end{enumerate}
%%

%%
Step 1, in classical imperative programming languages, is often made with
``loops''; when the declarative semantics is used, as in PROFETA, the
recursive approach is always needed.
%%
In our case, we model a number $X$ with the belief \texttt{number(X)} and,
to generate all numbers, we write the plans depicted below:
%%
\python%%
\begin{lstlisting}
+number("X") / (lambda : X < 100) >> [ "X = X + 1", +number("X") ]
+number("X") >> [ show_line("end of number generation") ]
\end{lstlisting}
%%
Plan in line 1 shows two new concepts of PROFETA: \emph{(i)} how to add
conditions on variables; and \emph{(ii)} how to use expressions to modify
variable values.
%%
Condition in line 1 is specified as a Python lambda function which has
\emph{no parameters} and returns a boolean value; therefore that plan is
executed as soon as the added belief \texttt{number()} has the parameter
value less than 100: if this is the case, the expression \texttt{"X = X +
  1"} is executed, which increments the parameter, and a new
\texttt{number()} belief is asserted.
%%
However, when the parameter reaches 100, plan in line 1 is no more
triggered and plan in line 2 can be executed thus showing a message.
%%

%%
These two plans implement the loop needed to generate our numbers which
is started by (manually) asserting belief\footnote{Yes, the number 1 is
  not present in the set of the Sieve of Eratosthenes, but we use it to
  trigger our program.} \texttt{number(1)}: indeed,
if we run the example with these two plans, it will behave as follows:
%%
\shellsnap{%
...\\
PROFETA>\underline{+number(1)}\\
PROFETA>end of number generation\\
\underline{kb}\\
{[number(1), number(2), number(3), number(4), number(5), number(6), number(7),\\
number(8), number(9), number(10), number(11), number(12), number(13), \\
number(14),number(15), number(16), number(17), number(18), number(19), \\
...\\
number(95), number(96), number(97), number(98), number(99), number(100)]}\\
PROFETA>}
%%

%%
It's now to time to execute step 2 of the Sieve; here the game is played
almost totally in the condition part of a plan: we must retrieve two
numbers, assure that they are \textbf{different}, and check if one is
multiple of the other one; if this is the case, the multiple must be
\textbf{removed} from the KB; the removal action can be also used to
\emph{retrigger} the process, thus implementing (once again) a
recursion-based loop.
%%
The plans that behave as explained are the following:
%%
\python%%
\begin{lstlisting}
-number("_") / (number("X") & number("Y") & \
               (lambda : X != Y) & \
               (lambda : (X % Y) == 0) ) \
               >> [ -number("X") ]
-number("_") >> [ show_line("prime numbers detected") ]
\end{lstlisting}
%%
The first thing to notice is the use of the special symbol \texttt{"\_"},
which, according to the Prolog syntax, means ``don't care'', i.e.~any
parameter value.
%%
Then the plan in lines 1--4 can be read as follows: when any
\texttt{number()} belief is removed, \emph{(i)} retrieve two other \texttt{number()}
beliefs---\verb+number("X") & number("Y")+; \emph{(ii)} check that the
numbers are different---\verb+(lambda : X != Y)+; \emph{(iii)} check if
\texttt{X} is a multiple of \texttt{Y}---\verb+(lambda : (X % Y) == 0)+;
then \emph{(iv)} remove \texttt{number(X)} since it is not prime; such a
removal retriggers the plan recursively.
%%
If the condition is false, all numbers are prime, so the plan in line 5 can
be executed showing the message.
%%

%%
The complete listing of the program is reported in Figure~\ref{fig:sieve}.
%%
There you can see the role of the belief \texttt{number(1)}, whose removal
is used to link the generation of numbers with prime number detection.
%%
Running the example is simple: asserting
\texttt{number(1)} in the PROFETA shell will start number generation and,
subsequently, prime number detection, so when the termination message
appears, the knowledge-base will contain only prime numbers.
%%

%%
\begin{figure}[b!]
\python%%
\begin{lstlisting}
from profeta.lib  import *
from profeta.main import *


class number(Belief):
    pass

PROFETA.start()

+number("N") / (lambda : N < 100) >> [ "N = N + 1", +number("N") ]
+number("_") >> [ show_line("end of sequence generation"),
                  -number(1) ]

-number("_") / (number("X") & \
                number("Y") & \
                (lambda : X != Y) & \
                (lambda : (X % Y) == 0) ) >> [ -number("X") ]
-number("_") >> [ show_line("prime numbers detected") ]

PROFETA.run_shell(globals())
\end{lstlisting}
\caption{The Sieve of Eratosthenes}\label{fig:sieve}
\end{figure}
%%
%%




%%
\section{Summary}
%%
In this Chapter, we have started to learn how to use PROFETA.
%%
We have understood its working scheme, introduced its syntax and an
informal semantics;
%%
we have learned how to use beliefs, in terms of their definition and
manipulation, and how beliefs can be exploited to perform reasoning by
means of reactive plans.
%%
The basic structure of a PROFETA program and its building blocks are then
introduced in order to let the reader to understand how to practically use
the tool and write simple programs.
%%



